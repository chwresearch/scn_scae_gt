% Options for packages loaded elsewhere
\PassOptionsToPackage{unicode}{hyperref}
\PassOptionsToPackage{hyphens}{url}
%
\documentclass[
]{article}
\usepackage{amsmath,amssymb}
\usepackage{lmodern}
\usepackage{iftex}
\ifPDFTeX
  \usepackage[T1]{fontenc}
  \usepackage[utf8]{inputenc}
  \usepackage{textcomp} % provide euro and other symbols
\else % if luatex or xetex
  \usepackage{unicode-math}
  \defaultfontfeatures{Scale=MatchLowercase}
  \defaultfontfeatures[\rmfamily]{Ligatures=TeX,Scale=1}
\fi
% Use upquote if available, for straight quotes in verbatim environments
\IfFileExists{upquote.sty}{\usepackage{upquote}}{}
\IfFileExists{microtype.sty}{% use microtype if available
  \usepackage[]{microtype}
  \UseMicrotypeSet[protrusion]{basicmath} % disable protrusion for tt fonts
}{}
\makeatletter
\@ifundefined{KOMAClassName}{% if non-KOMA class
  \IfFileExists{parskip.sty}{%
    \usepackage{parskip}
  }{% else
    \setlength{\parindent}{0pt}
    \setlength{\parskip}{6pt plus 2pt minus 1pt}}
}{% if KOMA class
  \KOMAoptions{parskip=half}}
\makeatother
\usepackage{xcolor}
\usepackage[margin=1in]{geometry}
\usepackage{color}
\usepackage{fancyvrb}
\newcommand{\VerbBar}{|}
\newcommand{\VERB}{\Verb[commandchars=\\\{\}]}
\DefineVerbatimEnvironment{Highlighting}{Verbatim}{commandchars=\\\{\}}
% Add ',fontsize=\small' for more characters per line
\usepackage{framed}
\definecolor{shadecolor}{RGB}{248,248,248}
\newenvironment{Shaded}{\begin{snugshade}}{\end{snugshade}}
\newcommand{\AlertTok}[1]{\textcolor[rgb]{0.94,0.16,0.16}{#1}}
\newcommand{\AnnotationTok}[1]{\textcolor[rgb]{0.56,0.35,0.01}{\textbf{\textit{#1}}}}
\newcommand{\AttributeTok}[1]{\textcolor[rgb]{0.77,0.63,0.00}{#1}}
\newcommand{\BaseNTok}[1]{\textcolor[rgb]{0.00,0.00,0.81}{#1}}
\newcommand{\BuiltInTok}[1]{#1}
\newcommand{\CharTok}[1]{\textcolor[rgb]{0.31,0.60,0.02}{#1}}
\newcommand{\CommentTok}[1]{\textcolor[rgb]{0.56,0.35,0.01}{\textit{#1}}}
\newcommand{\CommentVarTok}[1]{\textcolor[rgb]{0.56,0.35,0.01}{\textbf{\textit{#1}}}}
\newcommand{\ConstantTok}[1]{\textcolor[rgb]{0.00,0.00,0.00}{#1}}
\newcommand{\ControlFlowTok}[1]{\textcolor[rgb]{0.13,0.29,0.53}{\textbf{#1}}}
\newcommand{\DataTypeTok}[1]{\textcolor[rgb]{0.13,0.29,0.53}{#1}}
\newcommand{\DecValTok}[1]{\textcolor[rgb]{0.00,0.00,0.81}{#1}}
\newcommand{\DocumentationTok}[1]{\textcolor[rgb]{0.56,0.35,0.01}{\textbf{\textit{#1}}}}
\newcommand{\ErrorTok}[1]{\textcolor[rgb]{0.64,0.00,0.00}{\textbf{#1}}}
\newcommand{\ExtensionTok}[1]{#1}
\newcommand{\FloatTok}[1]{\textcolor[rgb]{0.00,0.00,0.81}{#1}}
\newcommand{\FunctionTok}[1]{\textcolor[rgb]{0.00,0.00,0.00}{#1}}
\newcommand{\ImportTok}[1]{#1}
\newcommand{\InformationTok}[1]{\textcolor[rgb]{0.56,0.35,0.01}{\textbf{\textit{#1}}}}
\newcommand{\KeywordTok}[1]{\textcolor[rgb]{0.13,0.29,0.53}{\textbf{#1}}}
\newcommand{\NormalTok}[1]{#1}
\newcommand{\OperatorTok}[1]{\textcolor[rgb]{0.81,0.36,0.00}{\textbf{#1}}}
\newcommand{\OtherTok}[1]{\textcolor[rgb]{0.56,0.35,0.01}{#1}}
\newcommand{\PreprocessorTok}[1]{\textcolor[rgb]{0.56,0.35,0.01}{\textit{#1}}}
\newcommand{\RegionMarkerTok}[1]{#1}
\newcommand{\SpecialCharTok}[1]{\textcolor[rgb]{0.00,0.00,0.00}{#1}}
\newcommand{\SpecialStringTok}[1]{\textcolor[rgb]{0.31,0.60,0.02}{#1}}
\newcommand{\StringTok}[1]{\textcolor[rgb]{0.31,0.60,0.02}{#1}}
\newcommand{\VariableTok}[1]{\textcolor[rgb]{0.00,0.00,0.00}{#1}}
\newcommand{\VerbatimStringTok}[1]{\textcolor[rgb]{0.31,0.60,0.02}{#1}}
\newcommand{\WarningTok}[1]{\textcolor[rgb]{0.56,0.35,0.01}{\textbf{\textit{#1}}}}
\usepackage{graphicx}
\makeatletter
\def\maxwidth{\ifdim\Gin@nat@width>\linewidth\linewidth\else\Gin@nat@width\fi}
\def\maxheight{\ifdim\Gin@nat@height>\textheight\textheight\else\Gin@nat@height\fi}
\makeatother
% Scale images if necessary, so that they will not overflow the page
% margins by default, and it is still possible to overwrite the defaults
% using explicit options in \includegraphics[width, height, ...]{}
\setkeys{Gin}{width=\maxwidth,height=\maxheight,keepaspectratio}
% Set default figure placement to htbp
\makeatletter
\def\fps@figure{htbp}
\makeatother
\setlength{\emergencystretch}{3em} % prevent overfull lines
\providecommand{\tightlist}{%
  \setlength{\itemsep}{0pt}\setlength{\parskip}{0pt}}
\setcounter{secnumdepth}{-\maxdimen} % remove section numbering
\ifLuaTeX
  \usepackage{selnolig}  % disable illegal ligatures
\fi
\IfFileExists{bookmark.sty}{\usepackage{bookmark}}{\usepackage{hyperref}}
\IfFileExists{xurl.sty}{\usepackage{xurl}}{} % add URL line breaks if available
\urlstyle{same} % disable monospaced font for URLs
\hypersetup{
  pdftitle={Memoria de Procesamiento COUs},
  pdfauthor={Renato Vargas},
  hidelinks,
  pdfcreator={LaTeX via pandoc}}

\title{Memoria de Procesamiento COUs}
\author{Renato Vargas}
\date{}

\begin{document}
\maketitle

\hypertarget{r-markdown}{%
\subsection{R Markdown}\label{r-markdown}}

This is an R Markdown document. Markdown is a simple formatting syntax
for authoring HTML, PDF, and MS Word documents. For more details on
using R Markdown see \url{http://rmarkdown.rstudio.com}.

When you click the \textbf{Knit} button a document will be generated
that includes both content as well as the output of any embedded R code
chunks within the document. You can embed an R code chunk like this:

\begin{Shaded}
\begin{Highlighting}[]
\FunctionTok{summary}\NormalTok{(cars)}
\end{Highlighting}
\end{Shaded}

\begin{verbatim}
##      speed           dist       
##  Min.   : 4.0   Min.   :  2.00  
##  1st Qu.:12.0   1st Qu.: 26.00  
##  Median :15.0   Median : 36.00  
##  Mean   :15.4   Mean   : 42.98  
##  3rd Qu.:19.0   3rd Qu.: 56.00  
##  Max.   :25.0   Max.   :120.00
\end{verbatim}

\hypertarget{including-plots}{%
\subsection{Including Plots}\label{including-plots}}

You can also embed plots, for example:

\includegraphics{MemoriaProcesamientoCOUs_files/figure-latex/pressure-1.pdf}

Note that the \texttt{echo\ =\ FALSE} parameter was added to the code
chunk to prevent printing of the R code that generated the plot.

\hypertarget{procesamiento-cuadros-de-oferta-y-utilizaciuxf3n---auxf1os-2013-2020}{%
\section{Procesamiento Cuadros de Oferta y Utilización - Años
2013-2020}\label{procesamiento-cuadros-de-oferta-y-utilizaciuxf3n---auxf1os-2013-2020}}

\hypertarget{introducciuxf3n}{%
\subsection{Introducción}\label{introducciuxf3n}}

Como primer paso llamamos las librerías necesarias para el procesamiento
de datos.

\begin{Shaded}
\begin{Highlighting}[]
\FunctionTok{library}\NormalTok{(readxl)}
\FunctionTok{library}\NormalTok{(openxlsx)}
\FunctionTok{library}\NormalTok{(reshape2)}
\FunctionTok{library}\NormalTok{(stringr)}
\FunctionTok{library}\NormalTok{(plyr)}
\FunctionTok{library}\NormalTok{(RSQLite)}
\FunctionTok{library}\NormalTok{(DBI)}
\FunctionTok{library}\NormalTok{(readr)}
\end{Highlighting}
\end{Shaded}

Limpiamos el área de trabajo para evitar problemas.

\begin{Shaded}
\begin{Highlighting}[]
\FunctionTok{rm}\NormalTok{(}\AttributeTok{list =} \FunctionTok{ls}\NormalTok{())}
\end{Highlighting}
\end{Shaded}

Es importante conocer la estructura de archivos de nuestro análisis y es
necesario mantener un mapa mental de los diferentes directorios en los
que tenemos nuestros datos y scripts de procesamiento. En el caso de
nuestro repositorio de análisis, nos interesa el directorio
\texttt{datos}.

\begin{verbatim}
scn_scae_gt
   ├───datos
   ├───documentos
   │   └───presentaciones
   └───scripts
\end{verbatim}

Empezamos por crear variables con datos básicos de nuestro análisis. El
nombre del país que estamos analizando, su código ISO de tres letras y
la ruta al archivo en donde se encuentran nuestros datos. Nótese que
normalmente si utilizamos un proyecto de RStudio, nuestra raíz se
encontrará en donde se encuentra el archivo \texttt{.RProj}. No
obstante, los documentos de Markdown tienen como raíz, el directorio en
el que se encuentra el archivo por lo que debemos hacer referencia a los
datos tomando esto en consideración.

\begin{Shaded}
\begin{Highlighting}[]
\NormalTok{pais }\OtherTok{\textless{}{-}} \StringTok{"Guatemala"}
\NormalTok{iso3 }\OtherTok{\textless{}{-}} \StringTok{"GTM"}
\NormalTok{archivo }\OtherTok{\textless{}{-}} \StringTok{"../../datos/GTM\_COUS\_DESAGREGADOS\_2013\_2020.xlsx"}
\end{Highlighting}
\end{Shaded}

Nuestro primer paso exploratorio será abrir el archivo de Excel y ver
con qué pestañas cuenta. Colocamos el listado de los nombres de las
pestañas en un vector llamado \texttt{hojas}.

\begin{Shaded}
\begin{Highlighting}[]
\NormalTok{hojas }\OtherTok{\textless{}{-}} \FunctionTok{excel\_sheets}\NormalTok{(archivo)}
\NormalTok{hojas}
\end{Highlighting}
\end{Shaded}

\begin{verbatim}
##  [1] "2013 C" "2014 C" "2014 K" "2015 C" "2015 K" "2016 C" "2016 K" "2017 C"
##  [9] "2017 K" "2018 C" "2018 K" "2019 C" "2019 K" "2020 C" "2020 K"
\end{verbatim}

La documentación del archivo de Excel nos indica que cada pestaña cuenta
con los cuadros de oferta y utilización para el año al que hace
referencia el nombre y estos pueden ser a precios corrientes si están
identificados con la letra ``C'' o precios constantes con la letra
``K''. El análisis de la cuenta de energía no utiliza los precios
constantes. Por conveniencia de análisis eliminamos las hojas de este
tipo de precios de nuestro listado a través de índices inversos pasando
al índice \texttt{{[}\ {]}}un signo menos seguido de un vector con las
posiciones a eliminar \texttt{-c()}.

\begin{Shaded}
\begin{Highlighting}[]
\NormalTok{hojas }\OtherTok{\textless{}{-}}\NormalTok{ hojas[}\SpecialCharTok{{-}}\FunctionTok{c}\NormalTok{(}\DecValTok{3}\NormalTok{,}\DecValTok{5}\NormalTok{,}\DecValTok{7}\NormalTok{,}\DecValTok{9}\NormalTok{,}\DecValTok{11}\NormalTok{,}\DecValTok{13}\NormalTok{,}\DecValTok{15}\NormalTok{)]}
\NormalTok{hojas}
\end{Highlighting}
\end{Shaded}

\begin{verbatim}
## [1] "2013 C" "2014 C" "2015 C" "2016 C" "2017 C" "2018 C" "2019 C" "2020 C"
\end{verbatim}

En el código a continuación se encuentran múltiples instancias de la
construcción \texttt{hojas{[}i{]}}. En vez de utilizar un número
absoluto (ej. \texttt{hojas{[}2{]}}) para hacer referencia a la posición
en el listado de la pestaña a analizar, utilizamos \texttt{i}, para
poder después reutilizar el código de manera recursiva para analizar
todas las hojas de manera programática, cambiando \texttt{i} por la
posición de la pestaña en cada iteración. No haremos esto en esta guía,
pero es un buen paso preparatorio.

\begin{Shaded}
\begin{Highlighting}[]
\NormalTok{i }\OtherTok{\textless{}{-}} \DecValTok{1}
\end{Highlighting}
\end{Shaded}

\hypertarget{procesamiento-inicial-del-archivo-de-excel}{%
\subsection{Procesamiento inicial del archivo de
Excel}\label{procesamiento-inicial-del-archivo-de-excel}}

Primeramente obtenemos información de referencia del archivo de Excel
que hemos inspeccionado visualmente en esa aplicación anteriormente.
Sabemos en qué celdas de cada cuadro se encuentra esa información.
Extraemos un rectángulo de celdas (una matriz) que cuenta con lo que nos
interesa.

\begin{Shaded}
\begin{Highlighting}[]
\NormalTok{info }\OtherTok{\textless{}{-}} \FunctionTok{read\_excel}\NormalTok{(}
\NormalTok{  archivo,}
  \AttributeTok{range =} \FunctionTok{paste}\NormalTok{(}\StringTok{"\textquotesingle{}"}\NormalTok{ , hojas[i], }\StringTok{"\textquotesingle{}!A8:B9"}\NormalTok{, }\AttributeTok{sep =} \StringTok{""}\NormalTok{) ,}
  \AttributeTok{col\_names =} \ConstantTok{FALSE}\NormalTok{,}
  \AttributeTok{col\_types =} \StringTok{"text"}\NormalTok{)}
\end{Highlighting}
\end{Shaded}

\begin{verbatim}
##      ...1                    ...2  
## [1,] "AÑO:"                  "2013"
## [2,] "Millones de quetzales" NA
\end{verbatim}

Y vemos dos datos importantes que colocamos en dos variables. Utilizando
la función str\_extract( ,``\textbackslash d\{4\}'') que combina la
expresión regular que indica que debe encontrar cuatro números juntos
obtenemos el año. Y la unidad de medida la obtenemos de la celda en la
segunda fila y primera columna.

\begin{Shaded}
\begin{Highlighting}[]
\NormalTok{anio }\OtherTok{\textless{}{-}} \FunctionTok{as.numeric}\NormalTok{((}\FunctionTok{str\_extract}\NormalTok{(info[}\DecValTok{1}\NormalTok{,}\DecValTok{2}\NormalTok{ ], }\StringTok{"}\SpecialCharTok{\textbackslash{}\textbackslash{}}\StringTok{d\{4\}"}\NormalTok{)))}
\NormalTok{unidad }\OtherTok{\textless{}{-}} \FunctionTok{toString}\NormalTok{(info[}\DecValTok{2}\NormalTok{,}\DecValTok{1}\NormalTok{ ]) }\CommentTok{\# unidad de medida}
\end{Highlighting}
\end{Shaded}

\begin{verbatim}
## [1] 2013
\end{verbatim}

\begin{verbatim}
## [1] "Millones de quetzales"
\end{verbatim}

Creamos un bucle para establecer otras variables de apoyo según tengamos
precios constantes o corrientes en nuestras variables iniciales.

\begin{Shaded}
\begin{Highlighting}[]
\CommentTok{\# Determinación de precios corrientes o constantes}
\ControlFlowTok{if}\NormalTok{ (unidad }\SpecialCharTok{!=} \StringTok{"Millones de quetzales"}\NormalTok{) \{}
\NormalTok{  precios }\OtherTok{\textless{}{-}} \StringTok{"Constantes"}
\NormalTok{  id\_precios }\OtherTok{\textless{}{-}} \DecValTok{2}
\NormalTok{  unidad }\OtherTok{\textless{}{-}} \FunctionTok{c}\NormalTok{(}\StringTok{"Millones de quetzales en medidas encadenadas de volumen con año de referencia 2013"}\NormalTok{)}
\NormalTok{  id\_unidad }\OtherTok{\textless{}{-}} \DecValTok{2}
\NormalTok{\}}\ControlFlowTok{else}\NormalTok{ \{}
\NormalTok{  precios }\OtherTok{\textless{}{-}} \StringTok{"Corrientes"}
\NormalTok{  id\_precios }\OtherTok{\textless{}{-}} \DecValTok{1}
\NormalTok{  id\_unidad }\OtherTok{\textless{}{-}} \DecValTok{1} \CommentTok{\# Millones de quetzales}
\NormalTok{\}}
\end{Highlighting}
\end{Shaded}

\begin{verbatim}
## [1] 1
\end{verbatim}

\begin{verbatim}
## [1] "Corrientes"
\end{verbatim}

\begin{verbatim}
## [1] "Millones de quetzales"
\end{verbatim}

El siguiente paso es empezar la ingesta de datos del cuadro de oferta.
Aprovechamos para establecer el identificador único del cuadro de
oferta:

\begin{Shaded}
\begin{Highlighting}[]
\NormalTok{id\_cuadro }\OtherTok{\textless{}{-}} \DecValTok{1} \CommentTok{\# 1  Oferta monetaria}
\end{Highlighting}
\end{Shaded}

Como puede verse en el parámetro range de la función read\_excel(), lo
que hacemos, en esencia, es leer todas las celdas numéricas del cuadro,
incluyendo espacios vacíos, columnas y filas de subtotales, para poder
clasificar cada celda respecto a cuatro dimensiones: la transacción a la
que corresponde, la actividad económica que la realiza, el producto
objeto de la transacción y una cuarta que captura los elementos no
estándar comunes en los cuadros de oferta y utilización publicados por
los bancos centrales e institutos de estadística a través de lo que
llamamos ``Áreas transaccionales de fila o de columna''.

Leemos todo el rectángulo que contiene datos.

\begin{Shaded}
\begin{Highlighting}[]
\NormalTok{oferta }\OtherTok{\textless{}{-}} \FunctionTok{as.matrix}\NormalTok{(}\FunctionTok{read\_excel}\NormalTok{(}
\NormalTok{  archivo,}
  \AttributeTok{range =} \FunctionTok{paste}\NormalTok{(}\StringTok{"\textquotesingle{}"}\NormalTok{ , hojas[i], }\StringTok{"\textquotesingle{}!C16:FL240"}\NormalTok{, }\AttributeTok{sep =} \StringTok{""}\NormalTok{),}
  \AttributeTok{col\_names =} \ConstantTok{FALSE}\NormalTok{,}
  \AttributeTok{col\_types =} \StringTok{"numeric"}
\NormalTok{))}
\end{Highlighting}
\end{Shaded}

Y en este punto simplemente le damos una identificación correlativa a
cada fila y a cada columna. En el caso de las filas, ambos cuadros de
oferta y utilización, comparten exactamente los mismos valores. Por esa
razón el correlativo solamente se compone de un identificador ISO3 de
país (GTM en este caso), una letra ``f'' minúscula (de filas) y un
correlativo de tres dígitos. Esto lo hacemos accediendo a los nombres de
fila y columna con las funciones \texttt{rownames()} y
\texttt{colnames()} y asignándoles la secuencia compuesta por los
elementos descritos.

Las funciones esenciales para lograr esto son \texttt{sprintf()}
combinada con `seq()'. Nótese que se embebe la función \texttt{paste()}
que nos permite concatenar el texto del código de país (GTM por ejemplo)
con el formato del correlativo de fila o columna. También embebemos la
función \texttt{dim()} dentro de la función \texttt{seq()} la cual nos
permite saber programáticamente las dimensiones de la matriz importada.

\begin{Shaded}
\begin{Highlighting}[]
\FunctionTok{rownames}\NormalTok{(oferta) }\OtherTok{\textless{}{-}} \FunctionTok{c}\NormalTok{(}\FunctionTok{sprintf}\NormalTok{(}\FunctionTok{paste}\NormalTok{(iso3,}\StringTok{"f\%03d"}\NormalTok{,  }\AttributeTok{sep =} \StringTok{""}\NormalTok{), }
                              \FunctionTok{seq}\NormalTok{(}\DecValTok{1}\NormalTok{, }\FunctionTok{dim}\NormalTok{(oferta)[}\DecValTok{1}\NormalTok{])))}
\FunctionTok{colnames}\NormalTok{(oferta) }\OtherTok{\textless{}{-}} \FunctionTok{c}\NormalTok{(}\FunctionTok{sprintf}\NormalTok{(}\FunctionTok{paste}\NormalTok{(iso3,}\StringTok{"oc\%03d"}\NormalTok{, }\AttributeTok{sep =} \StringTok{""}\NormalTok{), }
                              \FunctionTok{seq}\NormalTok{(}\DecValTok{1}\NormalTok{, }\FunctionTok{dim}\NormalTok{(oferta)[}\DecValTok{2}\NormalTok{])))}
\CommentTok{\# Y mostramos una parte para referencia}
\NormalTok{oferta[}\FunctionTok{c}\NormalTok{(}\DecValTok{1}\SpecialCharTok{:}\DecValTok{5}\NormalTok{),}\FunctionTok{c}\NormalTok{(}\DecValTok{1}\SpecialCharTok{:}\DecValTok{5}\NormalTok{)]}
\end{Highlighting}
\end{Shaded}

\begin{verbatim}
##            GTMoc001  GTMoc002 GTMoc003 GTMoc004 GTMoc005
## GTMf001    5.346939    0.0000        0        0        0
## GTMf002 4734.946097    0.0000        0        0        0
## GTMf003  101.132353    0.0000        0        0        0
## GTMf004    0.000000  659.4762        0        0        0
## GTMf005    0.000000 1539.4168        0        0        0
\end{verbatim}

El siguiente punto consiste en identificar las columnas a eliminar, las
cuales contienen subtotales y totales, contando como 1 la primera celda
leída en el módulo anterior. Esto lo hacemos inspeccionando visualmente
el archivo de Excel en esa aplicación previamente. El resultado es:

\begin{verbatim}
129 SIFMI (Lo eliminamos porque no tiene datos en ningún año)
130 P1 PRODUCCION (PB)  SUBTOTAL DE MERCADO
135 P1 PRODUCCION (PB)  SUBTOTAL USO FINAL PROPIO
147 P1 PRODUCCION (PB)  SUBTOTAL NO DE MERCADO
148 P1 TOTAL PRODUCCION (PB)
149 VACIA
150 VACIA
153 P7 IMPORTACIONES (CIF) TOTAL
155 TOTAL OFERTA (a precios básicos)
160 D21 IMPUESTOS SOBRE PRODUCTOS   TOTAL
165 MARGENES DE DISTRIBUCION TOTAL
166 TOTAL OFERTA (PC)   TOTAL OFERTA (PC)
\end{verbatim}

Seguidamnete, identificamos en nuestro Excel las filas vacías y que
contienen subtotales para ser eliminadas. Estas consisten del listado:

\begin{verbatim}
c(214, 215, 216, 218, 219, 224, 225)
\end{verbatim}

A través del concepto de índices {[}fila, columna{]}, utilizado en la
notación matricial para referirnos a la posición de una o más celdas en
una matriz, a través del signo menos (-) eliminamos las filas y columnas
identificadas en el paso anterior.

\begin{Shaded}
\begin{Highlighting}[]
\NormalTok{oferta }\OtherTok{\textless{}{-}}\NormalTok{ oferta[}\SpecialCharTok{{-}}\FunctionTok{c}\NormalTok{(}\DecValTok{214}\NormalTok{, }\DecValTok{215}\NormalTok{, }\DecValTok{216}\NormalTok{, }\DecValTok{218}\NormalTok{, }\DecValTok{219}\NormalTok{,}
                    \DecValTok{224}\NormalTok{, }\DecValTok{225}\NormalTok{),}
                 \SpecialCharTok{{-}}\FunctionTok{c}\NormalTok{(}\DecValTok{129}\NormalTok{, }\DecValTok{130}\NormalTok{,}\DecValTok{135}\NormalTok{,}\DecValTok{147}\NormalTok{,}\DecValTok{148}\NormalTok{,}\DecValTok{149}\NormalTok{,}
                    \DecValTok{150}\NormalTok{,}\DecValTok{153}\NormalTok{,}\DecValTok{155}\NormalTok{,}\DecValTok{160}\NormalTok{,}\DecValTok{165}\NormalTok{,}\DecValTok{166}\NormalTok{)]}
\end{Highlighting}
\end{Shaded}

El resultado es una matriz rectangular de datos válidos y únicos. Es
decir, sin subtotales y totales que duplican los datos básicos. El
siguiente paso transforma esa matriz de dos dimensiones a una tabla de
datos de una sola dimensión que nos va a permitir guardar y manipular
nuestros datos a través de una base de datos y el lenguaje de consultas
estructuradas SQL.

Esto se logra con la función \texttt{melt()} de la librería
\texttt{reshape2}. Nótese que para ahorrar un paso creamos columnas
adicionales que representan lo mínimo necesario para poder identificar
el año, el identificador de a qué cuadro pertenecen los datos y la
unidad de medida del cuadro.

Anteriormente, agregábamos un identificador de precios constantes o
corrientes. No obstante, en aras de estandarizar este procedimiento para
temas energéticos, ambientales de empleo y otros que no usan esa
diferenciación, decidimos que la distinción de constantes o corrientes
se hará a través del campo de unidad de medida \texttt{id\_unidad}. Es
decir Millones de quetzales constantes.

Por el momento no se incluirá precios constantes.

\begin{Shaded}
\begin{Highlighting}[]
\CommentTok{\# Unimos el resultado de melt() con columnas con los valores}
\CommentTok{\# de nuestras variables iniciales.}
\NormalTok{oferta }\OtherTok{\textless{}{-}} \FunctionTok{cbind}\NormalTok{(iso3, anio,id\_cuadro, }\FunctionTok{melt}\NormalTok{(oferta), id\_unidad)}

\CommentTok{\# Y renombramos según nuestra nomenclatura y convenciones}
\FunctionTok{colnames}\NormalTok{(oferta) }\OtherTok{\textless{}{-}}
  \FunctionTok{c}\NormalTok{(}\StringTok{"iso3"}\NormalTok{,}
    \StringTok{"anio"}\NormalTok{,}
    \StringTok{"id\_cuadro"}\NormalTok{,}
    \StringTok{"id\_fila"}\NormalTok{,}
    \StringTok{"id\_columna"}\NormalTok{,}
    \StringTok{"valor"}\NormalTok{,}
    \StringTok{"id\_unidad"}\NormalTok{)}

\CommentTok{\# Finalmente mostramos los primeros valores de la tabla}
\CommentTok{\# como referencia.}
\FunctionTok{head}\NormalTok{(oferta)}
\end{Highlighting}
\end{Shaded}

\begin{verbatim}
##   iso3 anio id_cuadro id_fila id_columna       valor id_unidad
## 1  GTM 2013         1 GTMf001   GTMoc001    5.346939         1
## 2  GTM 2013         1 GTMf002   GTMoc001 4734.946097         1
## 3  GTM 2013         1 GTMf003   GTMoc001  101.132353         1
## 4  GTM 2013         1 GTMf004   GTMoc001    0.000000         1
## 5  GTM 2013         1 GTMf005   GTMoc001    0.000000         1
## 6  GTM 2013         1 GTMf006   GTMoc001    0.000000         1
\end{verbatim}

\end{document}
